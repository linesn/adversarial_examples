
\documentclass[conference]{IEEEtran}
\usepackage{amssymb}
\usepackage{amsmath}

%%%%%%%%%%%%%%%%%%%%%%%%%%%%%%%%%%%%%%%%%%%%%%%%%%%%%%%%%%%%%%%%%%%%%%%%%%%%%%%%%%%%%%%%%%%%%%%%%%%%%%%%%%%%%%%%%%%%%%%%%%%%%%%%%%%%%%%%%%%%%%%%%%%%%%%%%%%%%%%%%%%%%%%%%%%%%%%%%%%%%%%%%%%%%%%%%%%%%%%%%%%%%%
\usepackage{amsfonts}

%TCIDATA{OutputFilter=LATEX.DLL}
%TCIDATA{Version=5.50.0.2960}
%TCIDATA{<META NAME="SaveForMode" CONTENT="1">}
%TCIDATA{BibliographyScheme=BibTeX}
%TCIDATA{Created=Wednesday, December 09, 2020 21:10:24}
%TCIDATA{LastRevised=Friday, December 11, 2020 17:24:56}
%TCIDATA{<META NAME="GraphicsSave" CONTENT="32">}
%TCIDATA{<META NAME="DocumentShell" CONTENT="Articles\SW\IEEE Transactions for Conferences">}
%TCIDATA{CSTFile=IEEEtran.cst}

\newtheorem{theorem}{Theorem}
\newtheorem{acknowledgement}[theorem]{Acknowledgement}
\newtheorem{algorithm}[theorem]{Algorithm}
\newtheorem{axiom}[theorem]{Axiom}
\newtheorem{case}[theorem]{Case}
\newtheorem{claim}[theorem]{Claim}
\newtheorem{conclusion}[theorem]{Conclusion}
\newtheorem{condition}[theorem]{Condition}
\newtheorem{conjecture}[theorem]{Conjecture}
\newtheorem{corollary}[theorem]{Corollary}
\newtheorem{criterion}[theorem]{Criterion}
\newtheorem{definition}[theorem]{Definition}
\newtheorem{example}[theorem]{Example}
\newtheorem{exercise}[theorem]{Exercise}
\newtheorem{lemma}[theorem]{Lemma}
\newtheorem{notation}[theorem]{Notation}
\newtheorem{problem}[theorem]{Problem}
\newtheorem{proposition}[theorem]{Proposition}
\newtheorem{remark}[theorem]{Remark}
\newtheorem{solution}[theorem]{Solution}
\newtheorem{summary}[theorem]{Summary}
\input{tcilatex}
\begin{document}

\title{Generating and defending against adversarial examples in
vision-optimized neural architectures}
\pubid{\copyright ~2020 }
\specialpapernotice{}
\date{December 9, 2020}

%TCIMACRO{%
%\TeXButton{Author Information}{\author{\authorblockN{Daniel Donoghue}
%\authorblockA{
%Email: ddonogh1@jhu.edu}
%\and
%\authorblockN{Nicholas Lines}
%\authorblockA{
%Email: nicholasalines@gmail.com}
%\and
%\authorblockN{Arnaldo Pereira}
%\authorblockA{
%Email: aepereira@gmail.com}
%}}}%
%BeginExpansion
\author{\authorblockN{Daniel Donoghue}
\authorblockA{
Email: ddonogh1@jhu.edu}
\and
\authorblockN{Nicholas Lines}
\authorblockA{
Email: nicholasalines@gmail.com}
\and
\authorblockN{Arnaldo Pereira}
\authorblockA{
Email: aepereira@gmail.com}
}%
%EndExpansion

%TCIMACRO{\TeXButton{Make Title}{\maketitle}}%
%BeginExpansion
\maketitle%
%EndExpansion

%TCIMACRO{\TeXButton{Begin abstract}{\begin{abstract}}}%
%BeginExpansion
\begin{abstract}%
%EndExpansion

As automated decision-making becomes more popular and more dependent upon
artificial intelligence, securing sensitive models from adversarial behavior
has become essential. Neural networks are particularly vulnerable to
so-called adversarial examples \cite{szegedy2014intriguing}, and various
attacks and defences have been explored in the literature.

Our intention in this paper is to demonstrate and confirm the results of
such attacks at an informative but modest scale. We apply two common attacks
to both the wide ResNet and GoogLeNet neural models, and test two defences,
in a reproducible computational environment. We show that significant
improvements in network robustness are available with minimal defence
measures.

The authors are listed alphabetically, and all made equal contributions.
This work is performed in association with the Johns Hopkins Engineering for
Professionals Program, as a project for EN.625.638.8VL2.FA20 Neural Networks.

All code and further reference materials are available online at
https://github.com/linesn/adversarial\_examples.

%TCIMACRO{\TeXButton{End abstract}{\end{abstract}}}%
%BeginExpansion
\end{abstract}%
%EndExpansion

%TCIMACRO{\TeXButton{Table of Contents}{\tableofcontents}}%
%BeginExpansion
\tableofcontents%
%EndExpansion

\bigskip

\section{Executive Summary}

Since 2014 when Szegedy et al \cite{szegedy2014intriguing} published the
first observation on the subject, adversarial examples have gained much
attention in both the study of adversarial machine learning research and the
more results-oriented world of practical neural archetecture, due to the
alarming weaknesses they expose and the interesting robustness that can be
introduced via defence efforts. The term "adversarial example" is used to
describe "an input to a machine learning model that is intentionally
designed to cause the model to make a mistake in its predictions, despite
resembling a valid input to a human" \cite{wiyatno2019adversarial}. As such
these examples are classed as evasion techniques by adversarial machine
learning theory, since their goal is to evade detection while producing
misinterpretations \cite{wiki:aml}. 

Most of the literature on the subject (in keeping with traditions in the
neural network community) uses image recognition tasks to demonstrate the
efficacy of attacks and defences, and we will do the same. In this paper we
will demonstrate successful us of the Fast Gradient Sign Method (FGSM)  \cite%
{goodfellow2014explaining} and Directed Gradient Sign (DGSM) \cite%
{madry2020adversarial} attacks against convolutional neural networks trained
with Imagenette data. We will then examine the results of applying two
common defences: first, perturbed prediciton averaging, and second, training
using adversarial examples. We confirm the observations of Goodfellow et al 
\cite{goodfellow2014explaining} and show that, while the Wide ResNet and
GoogLeNet architectures are very susceptible to the above attacks, the named
defences also produce significant improvement to the robustness of the
classifiers.  

\section{Project overview}

\subsection{Why are adversarial examples effective?}

In practice neural architectures based on linear components are prefered
(over, for example, radial basis components) because of their speed in
training and inference. However, it is this property that makes them
particularly vulnerable to the most common form of adversarial example \cite%
{goodfellow2014explaining}. Neural classifier inputs or features naturally
have some precision limit, such as a color range or pixel count, below which
perturbations are ignored. Consider an input vector $\mathbf{X}$, to which
we add a noise vector $\mathbf{\eta }$, where every $\eta \in \mathbf{\eta }$
is smaller than $\epsilon ,$ the precision limit. To humans and a first pass
review by machines, $\mathbf{X}$ and $\mathbf{X+\eta }$ are identical.
However, when the linear activity function is computed, the network's
weights are dotted with the input, yielding approximately 
\[
\mathbf{W}^{\intercal }(\mathbf{X+\eta )=W}^{\intercal }\mathbf{X+W}%
^{\intercal }\mathbf{\eta ,}
\]%
where the noise term $\mathbf{W}^{\intercal }\mathbf{\eta }$ can grow very
large if $\mathbf{W}$ is ill-conditioned. This means, in practice, that
networks reliant on linear activity functions can produce extremely
different outputs when given only minimally altered inputs. 

\subsection{Attacks}

The FGSM attack \cite{goodfellow2014explaining} takes advantage of this
weakness in a straightforward manner. The attacker forms the perturbation
vector $\mathbf{\eta }$ to match the cost function gradient sign for a given
input, computing 
\[
\mathbf{\eta }=\epsilon _{\ast }\text{sign}(\nabla _{\mathbf{X}}J(\mathbf{%
\theta },\mathbf{X},y))
\]%
where $\epsilon _{\ast }$ is the allowable level of perturbation, $J$ is the
network cost funtion, $\mathbf{\theta }$ is the vector of model parameters,
and $y$ is the true label. Thus, if the attacker is in posession of the
model and labeled training data, it is easy to train the network to behave
badly using simple backpropagation. The result is that the network will lose
certainty in the true label classification, and often misclassify the data
at random. The attack parameter $\epsilon _{\ast }$ may be scaled, of
course, but making $\epsilon _{\ast }$ much larger than the network
precision level $\epsilon $ may produce examples whose alteration is visible
to human reviewers, so smaller $\epsilon _{\ast }$ are desireable from the
adversarial perspective.

One can alter this attack to cause the network to favor a particular class
instead \cite{madry2020adversarial}. We will call this the Directed Gradient
Sign Method (DGSM). This time we use the loss function to direct the network
toward a specific desired label. We iterate using gradient descent for a
given number of iterations, and project the gradient onto the $l_{\infty }$%
-norm $\epsilon _{\ast }$-sphere, which has the effect of insisting that a
feature is not altered by more than $\epsilon _{\ast }$. For a given input $%
\mathbf{X}$, the attacker must solve the minimization problem 
\begin{eqnarray*}
\min_{\delta }\{J_{adv}(\mathbf{X}+\mathbf{\delta }) &=&J(\mathbf{\theta },%
\mathbf{X}+\mathbf{\delta },y_{desired}) \\
&&-J(\mathbf{\theta },\mathbf{X}+\mathbf{\delta },y_{true})\}\text{ } \\
\text{subject to }\left\vert \left\vert \mathbf{\delta }\right\vert
\right\vert _{\infty } &\leq &\epsilon _{\ast }
\end{eqnarray*}%
where $J$ is again the loss function, $\mathbf{\theta }$ the fixed network
parameters, $y_{desired}$ and $y_{true}$ are two different labels, and $%
\mathbf{\delta }$ is the directed perturbation vector. This requires using
forward passes and backpropagation within the network over $N$ iterations,
applying the update rule%
\begin{eqnarray*}
\mathbf{\delta }_{t} &=&\mathbf{\delta }_{t-1}-a\text{ sign}(\nabla L_{adv}(%
\mathbf{X}+\mathbf{\delta }_{t-1})), \\
\mathbf{\delta }_{t} &\leftarrow &\text{clip}(\mathbf{\delta }_{t},-\epsilon
_{\ast },\epsilon _{\ast })
\end{eqnarray*}%
begining with the zero vector $\mathbf{\delta }_{0}=\mathbf{0}$. The result
of this attack is that the classifier will incorrectly favor the chosen
label $y_{desired}$ in adversarial inputs, despite remaining perfectly
capable of correctly classifying unaltered inputs. 

\subsection{Defences}

A theme that has emerged in the literature is that there is a strong
correlation between generally robust networks and networks that are not
easily swayed by adversarial attacks. Of course, one also expects a cost in
effort or accuracy to be associated with increased robustness. 

Our first defence we test is simply Perturbed Prediction Averaging. This
defence has the advantage that it does not require retraining the network,
and the only increased expense is the cost of slower decisions at inference
time. We predict the class for each image based on an ensemble prediction
for the original image and $N-1$ additional perturbed versions of the image,
with the perturbations drawn uniformly from the $\epsilon _{\max }$-ball in
the $l_{\infty }$ sense around the image, choosing $\epsilon _{\max }$ to be
larger than any expected adversarial alteration level $\epsilon _{\ast }$.
For example, $\epsilon _{\max }$ can be set large enough that a uniform
random $\left\vert \left\vert \mathbf{\delta }\right\vert \right\vert
_{\infty }\leq $ $\epsilon _{\max }$ perturbation would be easily noticed by
a human. In that case, we can assume that adversarial attacks will rely on $%
\epsilon _{\ast }<<\epsilon _{\max }$. Using a modified softmax function, we
can express the probability for class $k$ of classes $\{1,2,...,K\}$
predicted using this defense as%
\[
P(y_{k}\rvert \mathbf{X})=\frac{\sum_{i=1}^{N}e^{z_{k}}(\mathbf{X}+\mathbf{%
\delta }_{i}\mathbf{)}}{\sum_{j=1}^{K}\sum_{i=1}^{N}e^{z_{j}}(\mathbf{X}+%
\mathbf{\delta }_{i}\mathbf{)}},
\]%
where $z_{j}(\cdot )$ is the output of the $j$th hidden node for a given
network input $\mathbf{X}$ and with $\mathbf{\delta }_{i}=\mathbf{0}$, and $%
\left\vert \left\vert \mathbf{\delta }\right\vert \right\vert _{\infty }\leq
\epsilon _{\max }$ for all $i$.

\section{Computational results}

\subsection{Creating Adversarial Examples}

\subsection{Defences}

\section{Analysis}

\section{Conclusions}

\bibliographystyle{amsplain}
\bibliography{acompat,JHU}

\end{document}
