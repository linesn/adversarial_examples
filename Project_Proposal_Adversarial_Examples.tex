%\theoremstyle{plain}


\documentclass{amsproc}
%%%%%%%%%%%%%%%%%%%%%%%%%%%%%%%%%%%%%%%%%%%%%%%%%%%%%%%%%%%%%%%%%%%%%%%%%%%%%%%%%%%%%%%%%%%%%%%%%%%%%%%%%%%%%%%%%%%%%%%%%%%%%%%%%%%%%%%%%%%%%%%%%%%%%%%%%%%%%%%%%%%%%%%%%%%%%%%%%%%%%%%%%%%%%%%%%%%%%%%%%%%%%%%%%%%%%%%%%%%%%%%%%%%%%%%%%%%%%%%%%%%%%%%%%%%%
\usepackage{amsfonts}

\setcounter{MaxMatrixCols}{10}
%TCIDATA{OutputFilter=LATEX.DLL}
%TCIDATA{Version=5.50.0.2960}
%TCIDATA{<META NAME="SaveForMode" CONTENT="1">}
%TCIDATA{BibliographyScheme=Manual}
%TCIDATA{Created=Friday, October 16, 2020 13:19:19}
%TCIDATA{LastRevised=Wednesday, November 11, 2020 15:51:28}
%TCIDATA{<META NAME="GraphicsSave" CONTENT="32">}
%TCIDATA{<META NAME="DocumentShell" CONTENT="Articles\SW\AMS Proceedings Article">}
%TCIDATA{Language=American English}
%TCIDATA{CSTFile=nl-amsprtci.cst}
%TCIDATA{ComputeDefs=
%$g=p-1$
%$x_{1}=3$
%$x_{2}=4$
%}


\theoremstyle{definition}
\newtheorem{acknowledgement}{Acknowledgement}[section]
\newtheorem{algorithm}{Algorithm}[section]
\newtheorem{axiom}{Axiom}[section]
\newtheorem{case}{Case}[section]
\newtheorem{claim}{Claim}[section]
\newtheorem{conclusion}{Conclusion}[section]
\newtheorem{condition}{Condition}[section]
\newtheorem{conjecture}{Conjecture}[section]
\newtheorem{corollary}{Corollary}[section]
\newtheorem{criterion}{Criterion}[section]
\newtheorem{definition}{Definition}[section]
\newtheorem{example}{Example}[section]
\newtheorem{exercise}{Exercise}[section]
\newtheorem{lemma}{Lemma}[section]
\newtheorem{notation}{Notation}[section]
\newtheorem{problem}{Problem}[section]
\newtheorem{proposition}{Proposition}[section]
\newtheorem{remark}{Remark}[section]
\newtheorem{solution}{Solution}[section]
\newtheorem{summary}{Summary}[section]
\newtheorem{theorem}{Theorem}[section]
\numberwithin{equation}{section}
\input{tcilatex}

\begin{document}
\title[Adversarial Examples]{Neural Nets Project Proposal: Adversarial
Examples}
\author{Daniel Donoghue}
\email{ddonogh1@jhu.edu}
\author{Nicholas Lines}
\email{nicholasalines@gmail.com}
\author{Arnaldo Pereira}
\email{aepereira@gmail.com}
\date{November 11, 2020}

\begin{abstract}
Adversarial approaches and associated model defences are an important area
of machine learning research and practice. Neural networks are particularly
vulnerable to so-called adversarial examples \cite{szegedy2014intriguing},
which has led many to experiment with techniques for defeating these attacks
and make networks more robust in general. 

Our project will explore this issue using the well-known ImageNet dataset to
both produce and attempt to defeat adversarial examples created in standard
ways, using one or more neural archetectures. This will familiarize the
researchers and other course participants with current issues in machine
learning security in general and neural network robustness in particular,
while providing an opportunity to review and apply architectures discussed
or referenced in our course.

The authors are listed alphabetically, and all made equal contributions.
This work is performed in association with the Johns Hopkins Engineering for
Professionals Program, as a project for EN.625.638.8VL2.FA20 Neural Networks.
\end{abstract}

\maketitle

\section{Introduction and background}

As neural networks and other forms of machine learning have become
ubiquitous in many areas of industry, academic research, and government use,
questions related to machine learning security have come into sharper focus
and gathered significant attention. This is evidenced by the sheer volume of
papers submitted and accepted at data science and machine learning
conferences such as the International Conference on Machine Learning and
Neurips. 

With this field of research has come a new vocabulary, so let us first
establish a handful of useful terms in roughly the vernacular of modern
research \cite{wiki:aml}. Adversarial machine learning refers to the misuse
of data, models, and algorithms related to machine learning tasks, with the
intent to defeat or exploit these elements in a manner unintended by their
author\footnote{%
It is important for those new to this field of study to distinguish it from
the unfortunately similarly-named area of adversarial networks such as GANs,
which employ two opposing computational entities (i.e. separate networks) to
generate data and discriminate the quality of the data, in an effort to
match a desired set of features. }. These attacks generally fall into one of
the following categories.

\bigskip 

\begin{enumerate}
\item \textbf{Evasion methods:} Without access to the original training
data, an adversary attempts to make data that is misclassified or
misunderstood by the network. Usually the network is viewed as a black box
that the adversary wishes to fool, by guessing features it will care about.
As an example, consider pirates who upload copyrighted material that has
been inverted to YouTube to try to defeat anti-piracy reviews.

\item \textbf{Data poisoning:} In this scenario the adversary has influence
over the training data used to create or refine a model, and they exercise
this to introduce data that will produce misclassifications or other
incorrect model decisions. The classic example is the introduction of a
mislabeled stop sign with a minor alteration in an autonomous vehicle's
training data, which may lead the vehicle to interpret a similarly altered
stop sign in real-life as a speed limit sign. The adversary may or may not
have a transparent view of the model.

\item \textbf{Extraction methods:} This category covers attempts by
adversaries to extract from interactions with \ a model something they
weren't intended to have, usually some or all of the training data, or the
model itself. An example of this might be a competitor probing an
image-recognition application to recreate the model cheaply and reuse it in
another illegitimate setting. 
\end{enumerate}

Each of these approaches has a role in neural network research, but we will
restrict our scope to discussing evasion and poisoning methods used to trick
neural networks into making incorrect classification decisions. 

This area of work is about 6 years old. In 2014 Szegedy et al discovered
what they called "intriguing properties of neural networks," essentially
that, despite making classification decisions that were extremely highly
correlated with those of human experts, the machine-made and human-made
decisions used entirely different justifications to reach the same
conclusions. This implies that it is all too easy for an adversary to
produce new synthetic data that the human and machine judges will
unexpectedly classify entirely differently. The classic example of a panda
that GoogLeNet mistakes for a gibbon thanks to added noise that is invisible
to humans is shown in Fig. \ref{panda}. This example was part of Goodfellow
et al's first of many papers \cite{goodfellow2014explaining} on the subject
of creating, detecting, and defeating such adversarial examples.

\FRAME{ftbpFU}{5.9084in}{2.4102in}{0pt}{\Qcb{A miscategorized panda,
reproduced from \protect\cite{goodfellow2014explaining}.}}{\Qlb{panda}}{%
panda.jpg}{\special{language "Scientific Word";type
"GRAPHIC";maintain-aspect-ratio TRUE;display "USEDEF";valid_file "F";width
5.9084in;height 2.4102in;depth 0pt;original-width 5.8487in;original-height
2.3696in;cropleft "0";croptop "1";cropright "1";cropbottom "0";filename
'panda.jpg';file-properties "XNPEU";}}

\section{Project description}

We propose an exploration of the generation and defeat of adversarial
examples applied to an image classification task using the standard ImageNet
dataset \cite{imagenet_cvpr09}. Our work will be performed in Python using
standard libraries where possible, and publicly available under a standard
MIT\ license on GitHub\footnote{%
https://github.com/linesn/adversarial\_examples}. The goal of this project
is to create, apply, and attempt to defeat one or more adversarial examples
in the context of one or more neural architectures. In the interest of
allowing a naturally expandable scope, we leave open for the moment the
question of how many architectures and defeating techniques we will use,
stating only that we will use at least one of each, sampled from the
following lists.

\bigskip 

\begin{tabular}{ll}
\textbf{Network architectures of interest} & \textbf{Defeat methods} \\ 
(Multilayered) Perceptrons & Dropout (regularization) \\ 
Convolutional Neural Networks & Pruning \\ 
Variations with different learning methods & Training using perturbed
training data \\ 
& Batch normalization%
\end{tabular}

\bigskip 

We will deliver, as a group, at least one successful adversarial example
(that is, an attack that worked)\ against at least one network, attempt to
defeat the attack, iterate onward through other defeat methods and
architectures as time allows. 

This project is not structured as a novel scientific experiment, but rather
as a confirmation of scientific theory. While our particular results will
never have been shown before, they should resemble expected behavior that
has been discussed by neural defence researchers for the past five years. If
on the other hand we are unable to produce or defeat these previously
discovered attacks, we will analyze why our experiment differed from the
accepted model.

\section{Timeline}

As we have not yet discussed in our course a timeline for the project, we
propose the following.

\bigskip 

\begin{tabular}{ll}
\textbf{Date} & \textbf{Deliverable} \\ 
11/11/2020 & First draft of proposal \\ 
12/01/2020 & Preliminary results documented in a presentation to give in
class \\ 
12/12/2020 & Project paper turned in%
\end{tabular}

\section{Concluding thoughts}

We hope this project will not only advance the authors's understanding of
adversarial machine learning's intersection with neural networks, but also
serve as an introduction to the subject for our classmates. As the world
becomes more and more dependent upon real-time processing of big data using
artificial intelligence and other machine-made decisions, gaining an
appreciation and intuition for security and reliability of models and
algorithms will be of primary importance to machine learning practitioners
and researchers alike.

\newpage 

\bibliographystyle{IEEEtran}
\bibliography{acompat,JHU}

\end{document}
